\documentclass{article}

\usepackage{amssymb}
\usepackage{amsthm}
\usepackage{amsmath}

\usepackage[utf8x]{inputenc}
\usepackage[T1]{fontenc}
\usepackage[polish]{babel}
\usepackage{multicol}
\usepackage{graphicx}
 \usepackage{tikz}
 \usetikzlibrary{shapes,backgrounds}
\begin{document}
	
	
	
	Firma produkuje $98$\% wyrobów odpowiadających normie. Wśród wyrobów spełniających normę jest $75$\% wyrobów pierwszego gatunku. Obliczyć prawdopodobieństwo, że losowo wybrany wyrób jest pierwszego gatunku. \\* \\*		
	B - przeszedl norme  98\%  \newline
	A - zostal wybrany 1 gatunek \newline
	P(A) = ?
	
	$$ P(A|B) = 0,75 $$
	
	$$ P(A|B) = \dfrac{P(A \cap B)}{ P(B)} $$
	
	Stad
	
	$$ 0,75 = \dfrac{P(A \cap B)}{0,98} $$
	
	$$ P(A) = P(A \cap B) = 0,75 * 0, 98 = 0,735 $$
	
	
	
	
	
	
	
\end{document}
