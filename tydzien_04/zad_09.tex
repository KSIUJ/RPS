\subsection{Zadanie 9}


Gęstość zmiennej losowej $X$ o rozkładzie wykładniczym z parametrem $\lambda$ jest dana przez 
$$
f(x) = \left\{ \begin{array}{ll}
\lambda \cdot e^{-\lambda x} & \textrm{gdy $x > 0$}\\
0 & \textrm{gdy $x \leq 0$}
\end{array} \right.
$$

Zatem jeżeli za parametr  $\lambda$  przyjmiemy 1 otrzymamy:

$$
f(x) = \left\{ \begin{array}{ll}
 e^{- x} & \textrm{gdy $x > 0$}\\
0 & \textrm{gdy $x \leq 0$}
\end{array} \right.
$$
  
Co przedstawione zostało na rysunku. Prawdopodobieństwo, że  $X \in [0,1]$ zostało przedstawione na rysunku poprzez obszar zakreskowany na czerwono.

Obliczmy prawdopodobienstwo zdarzenia że $X \in [0,1]$:

$$
\int_{0}^{1} e^{- x} dx = [-e^x]_{0}^{1} = -e^{-1} + 1
$$
