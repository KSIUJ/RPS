\subsection{Zadanie 3}

Firma zakupiła $4$ nowe monitory tej samej marki. Prawdopodobieństwo, że monitor tej
marki ulegnie awarii w okresie gwarancji wynosi $0,05$. Oblicz prawdopodobieństwo, że

\begin{enumerate}[label=(\alph*)]
\item dwa monitory ulegną awarii w okresie gwarancji

$$
P(X=2) = \frac{4}{2} * 0.05^{2}* 0.95^{2}
$$

\item nie wszystkie monitory ulegną awarii w okresie gwarancji
$$
P(X<4)=1-P(X=4)=1- \frac{4}{4} * 0.05^{4} * 0.05^{0}
$$


\item co najmniej jeden monitor ulegnie awarii w okresie gwarancji
$$
P(X>0)=1- P(X=0)=1- \frac{4}{0} * {0.05}^0 * {0.35}^4
$$
\end{enumerate}

Czy te zdarzenia można opisać za pomocą jednej zmiennej losowej czy potrzebujemy różnych? 
Policz dla wyznaczonych zamiennych losowych wartość oczekiwaną i wariancję.

$$ 
EX=4*{0.05}
$$

$$
Var(X)=4*{0.05}*{0.95}
$$
