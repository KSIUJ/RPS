\subsection{Zadanie 10}

Z warunków zadania wynika, że dysponujemy próbką prostą $x_{1},...,x_{n}$ z rozkładu ciągłego, 
którego gęstość $f$ jest następująca:
$$
f(x) =
\left\{ \begin{array}{ll}
\frac{1}{a} & \text{gdy } 0 \leq x \leq a \\
0 & \text{wpp}
\end{array} \right.
$$
Funkcją wiarygodności jest więc tutaj:
$$
l(a) = f(x_{1}) \cdot ... \cdot f(x_{n})
$$
W związku z tym, jeżeli wszystkie punkty $x_{i}$ leżą w przedziale $(0, a)$, to:
$$
l(a) = \frac{1}{a^{n}}
$$
Zaś w przeciwnym przypadku:
$$
l(a) = 0
$$
Zatem:
$$
l(a) = 
\left\{ \begin{array}{ll}
\frac{1}{a^n} & \text{dla } a \geq max\{x_{1},...,x_{n}\} \\
0 & \text{dla } 0 < a < max\{x_{1},...,x_{n}\}
\end{array} \right.
$$
W nietrywialnym przypadku, czyli gdy $max\{x_{1},...,x_{n}\} > 0$ funkcja ta jest dobrze określona,
lecz nie jest ciągła w punkcie $ a = max\{x_{1},...,x_{n}\}$. \\
Jednak widać, że akurat w tym punkcie funkcja $l$ przyjmuje wartość największą. \\
Tak więc estymatorem największej wiarygodności parametru $a$ jest:
$$
\hat{a} = max\{x_{1},...,x_{n}\}
$$
