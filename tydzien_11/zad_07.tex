\subsection{zadanie 7}
W centrali telefonicznej dokonano 17 obserwacji długości losowo wybranych rozmów w ciągu jednego dnia i otrzymano (w min): $\bar X = 5.48$, $S=1.2$; na tej podstawie -- przy założeniu, że długość rozmów telefonicznych mają rozkład normalny -- wyznaczyć 95\%-ową realizację przedziału ufności dla wartości przeciętnej długości rozmowy telefonicznej przeprowadzonej za pośrednictwem tej centrali w danym dniu.


\textbf{\underline{Dane:}} \ \ $n = 17$,
$\alpha = 0.05$

\textbf{\underline{Wzór, z którego będę korzystać:}}
$$
\left(
\bar X - t \left(1-\frac{\alpha}{2},n-1 \right) \cdot \frac{S}{\sqrt{n-1}},
\bar X + t \left(1-\frac{\alpha}{2},n-1 \right) \cdot \frac{S}{\sqrt{n-1}}
\right)
$$

\textbf{\underline{Rozwiązanie:}}

\begin{center}
\begin{tabular}{ |c|c|c| } 
\hline
& &\\
& Lewe domknięcie przedziału & Prawe domknięcie przedziału \\ 
& & \\ \hline
& & \\
Wzór: & \large{$L = \bar X - t \left(1-\frac{\alpha}{2},n-1 \right) \cdot \frac{S}{\sqrt{n-1}}$} & \large{$P = \bar X + t \left(1-\frac{\alpha}{2},n-1 \right) \cdot \frac{S}{\sqrt{n-1}}$} \\
& & \\ \hline
& & \\
& $L = 5.48 - t \left(1-\frac{0.05}{2}, 16 \right) \cdot \frac{1.2}{\sqrt{16}} =$ & $P = 5.48 + t \left(1-\frac{0.05}{2}, 16 \right) \cdot \frac{1.2}{\sqrt{16}} = $ \\
& & \\
& $= 5.48 - t \left(1-0.025, 16 \right) \cdot \frac{1.2}{4} =$ & $= 5.48 + t \left(1-0.025, 16 \right) \cdot \frac{1.2}{4} = $ \\
& & \\
Rozwiązanie: & $= 5.48 - t \left(0.975, 16 \right) \cdot 0.3 =$ & $= 5.48 + t \left(0.975, 16 \right) \cdot 0.3 = $ \\
& & \\
 & $= 5.48 - 0.128 \cdot 0.3 =$ & $= 5.48 + 0.128 \cdot 0.3 = $ \\
& & \\
 & $= 5.48 - 0.0384 = 5.4416$ & $= 5.48 + 0.0384 = 5.5184$ \\
& & \\ \hline
\end{tabular}
\newline
\begin{tabular}{ |c| } 
\hline
\\
Wartość kwantyla rozkładu t-Studenta odczytuje z tablic: \\
$ t \left(0.975, 16 \right) = 0.128$ \\
\\ \hline
\end{tabular}
\end{center}

\textbf{\underline{Odpowiedź:}} \Large{Szukanym przedziałem ufności jest przedział: \textbf{$\left(5.4416, 5.5184 \right)$}.}
\newline
