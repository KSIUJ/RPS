\subsection{Zadanie 9}
W losowo wybranej grupie 10 samochodów osobowych przeprowadzono badanie zużycia benzyny na -- tej samej dla wszystkich samochodów -- trasie długości 100 km. Okazało się, że średnia zużycia benzyny (w 1/100 km) dla tej grupy samochodów wynosi $\bar X = 8.1$; odchylenie standardowe $0,8$. Zakładając, że badana cecha ma rozkład normalny, wyznaczyć 99\%-ową realizację przedziału ufności dla wartości przeciętnej zużycia benzyny przez samochody tej marki na rozpatrywanej trasie.  


\textbf{\underline{Dane:}}
$n = 10$, 
$\alpha = 0.05$,
$\bar X = 8,1$,
$ \alpha = 0,01$,
$\sigma = 0,8$


Korzystam z modelu I bo mamy odchylenie standardowe.

$$ ( 8,1-u(0,995)*\frac{0,8}{\sqrt{10}} , 8,1 + u(0,955)*\frac{0,8}{\sqrt{10}} ) $$

$$ u(0,995) = 2,5758  $$

$$ (8,1 - \frac{2,5758*0,8}{\sqrt{10}}, 8,1 + \frac{2,5758*0,8}{\sqrt{10}}) $$

Rozwiązanie:
\\

$$ (7,44837   ,   8,75163)  $$
