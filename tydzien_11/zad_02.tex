\subsection{zadanie 2}
Zmierzono wytrzymałość 10 losowo wybranych gotowych elementów konstrukcji budowlanych i otrzymano następujące wyniki: 383, 284, 339, 340, 305, 386, 378, 335, 344, 346.

Zakładamy, że rozkład wytrzymałości tych elementów jest rozkładem normalnym $N(\mu,\sigma^2)$ o nieznanych parametrach. Wyznaczyć na podstawie tej próbki 95\%--ową realizację przedziału ufności dla nieznanej wartości parametru $\mu$ badanej cechy populacji. 
\newline

\textbf{\underline{Dane:}}
$n = 10$,
$\alpha = 0.05$
\newline

\textbf{\underline{Wzór, z którego będę korzystać:}}
$$
\left(
\bar X - t \left(1-\frac{\alpha}{2},n-1 \right) \cdot \frac{S}{\sqrt{n-1}},
\bar X + t \left(1-\frac{\alpha}{2},n-1 \right) \cdot \frac{S}{\sqrt{n-1}}
\right)
$$

\textbf{\underline{Rozwiązanie:}}

\begin{flushleft} 
\large{$\bar X = \frac{383+284+339+340+305+386+378+335+344+346}{10} =\frac{3440}{10} = 344$} \newline \newline
\large{$S^{2} = \frac{1}{10} \cdot \sum_{i=1}^{10} (X_{i} - 344) = \frac{39^{2}+(-60)^{2}+(-5)^{2}+(-39)^{2}+42^{2}+34^{2}+(-9)^{2}+2^{2}}{10} = \frac{9688}{10} = 968.8$} \newline \newline
$S = \sqrt{S^{2}}  = \sqrt{968.8} \approx 31.13$
\end{flushleft}

\begin{center}
\begin{tabular}{ |c| c | c| } 
\hline
& & \\
& Lewe domknięcie przedziału & Prawe domknięcie przedziału \\ 
& & \\ \hline
& & \\
Wzór & $L = \bar X - t \left(1-\frac{\alpha}{2},n-1 \right) \cdot \frac{S}{\sqrt{n-1}}$ & $P = \bar X + t \left(1-\frac{\alpha}{2},n-1 \right) \cdot \frac{S}{\sqrt{n-1}}$ \\
& & \\\hline
& & \\
& $L = 344 - t \left(1-\frac{0.05}{2}, 9 \right) \cdot \frac{31.13}{\sqrt{9}} = $ &  $P = 344 + t \left(1-\frac{0.05}{2}, 9 \right) \cdot \frac{31.13}{\sqrt{9}} = $  \\
& $= 344 - t \left(1-0.025, 9 \right) \cdot \frac{31.13}{3} = $ & $= 344 + t \left(1-0.025, 9 \right) \cdot \frac{31.13}{3} = $  \\
Rozwiązanie: & $= 344 - t \left(0.975, 9 \right) \cdot 10.38 = $ & $= 344 + t \left(0.975, 9 \right) \cdot 10.38 = $ \\
& $= 344 - 0.129 \cdot 10.38 = $ & $= 344 + 0.129 \cdot 10.38 = $ \\
& $= 344 - 1.34 = 342.66$ & $= 344 + 1.34 = 345.34$ \\

& & \\\hline
\end{tabular}
\newline \newline
\begin{tabular}{ |c c| } 
\hline
&  \\
Wartość rozkładu t-Studenta odczytuję z tablic: & \\

$t(0.975, 9) = 0.129$ & \\
& \\ \hline
\end{tabular}
\end{center}
\textbf{\underline{Odpowiedź:}} \Large{Szukanym przedziałem ufności jest przedział: \textbf{$\left( 342.66, 345.34 \right)$}.}
