\subsection{zadanie 8}
Z grupy robotników pewnego zakładu wykonujących taką samą pracę wybrano w sposób losowy 13 pracowników i dokonano badani pod względem wydajności pracy (w szt./h) uzyskując dane: 21, 12, 11, 15, 9, 10, 17, 8, 16, 13, 12, 9, 18. Na tej podstawie zakładając, że badana cech ma rozkład normalny, wyznaczyć 95\%-ową realizację przedziału ufności dal nieznanej wartości przeciętnej wydajności pracy.
\newline

\textbf{\underline{Dane:}}
$n = 13$,
$\alpha = 0.05$
\newline

\textbf{\underline{Wzór, z którego będę korzystać:}}
$$
\left(
\bar X - t \left(1-\frac{\alpha}{2},n-1 \right) \cdot \frac{S}{\sqrt{n-1}},
\bar X + t \left(1-\frac{\alpha}{2},n-1 \right) \cdot \frac{S}{\sqrt{n-1}}
\right)
$$

\textbf{\underline{Rozwiązanie:}}

\begin{flushleft} 
\large{$\bar X = \frac{21+12+11+15+9+10+17+8+16+13+12+9+18}{13} =\frac{171}{13} \approx 13,15$} \newline \newline
\large{$S^{2} = \frac{1}{13} \cdot \sum_{i=1}^{13} (X_{i} - 13,15) = \frac{(7,85)^{2}+(-1,15)^{2}+(-2,15)^{2}+(1,85)^{2}+(-4,15)^{2}+(-3, 15)^{2}+(3,85)^{2}+(-5,15)^{2}+(2,85)^{2}+(-0,15)^{2}+(-1,15)^{2}+(-4,15)^{2}+(4,85)^{2}}{13} = \frac{189,6925}{13} \approx 14,59$} \newline \newline
$S = \sqrt{S^{2}}  = \sqrt{14,59} \approx 3,82$
\end{flushleft}

\begin{center}
\begin{tabular}{ |c| c | c| } 
\hline
& & \\
& Lewe domknięcie przedziału & Prawe domknięcie przedziału \\ 
& & \\ \hline
& & \\
Wzór & $L = \bar X - t \left(1-\frac{\alpha}{2},n-1 \right) \cdot \frac{S}{\sqrt{n-1}}$ & $P = \bar X + t \left(1-\frac{\alpha}{2},n-1 \right) \cdot \frac{S}{\sqrt{n-1}}$ \\
& & \\\hline
& & \\
& $L = 13,15 - t \left(1-\frac{0.05}{2}, 12 \right) \cdot \frac{3,82}{\sqrt{12}} = $ &  $P = 13,15 + t \left(1-\frac{0.05}{2}, 12 \right) \cdot \frac{3,82}{\sqrt{12}} = $  \\
& $= 13.15 - t \left(1-0.025, 12 \right) \cdot \frac{3.82}{3.46} = $ & $= 13.15 + t \left(1-0.025, 12 \right) \cdot \frac{3.82}{3.46} = $  \\
Rozwiązanie: & $= 13.15 - t \left(0.975, 12 \right) \cdot 1.10 = $ & $= 13.15 + t \left(0.975, 12 \right) \cdot 1.10 = $ \\
& $= 13.15 - 0.128 \cdot 1.10 = $ & $= 13.15 + 0.128 \cdot 1.10 = $ \\
& $= 13.15 - 0.14 = 13.01$ & $= 13.15 + 0.14 = 13.29$ \\

& & \\\hline
\end{tabular}
\newline \newline
\begin{tabular}{ |c c| } 
\hline
&  \\
Wartość rozkładu t-Studenta odczytuję z tablic: & \\
$t(0.975, 12) = 0.128$ & \\
& \\ \hline
\end{tabular}
\end{center}
\textbf{\underline{Odpowiedź:}} \Large{Szukanym przedziałem ufności jest przedział: \textbf{$\left( 13.01, 13.29\right)$}.}
