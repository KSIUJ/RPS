\subsection{Zadanie 13}

X jest zmienną losową o rozkładzie jednostajnym na przedziale $[-2,2]$

$$F_Y(t)=P(Y \leq t)=P(X^2 \leq t)=P(|X| \leq \sqrt t)=\\
P(X \in [- \sqrt t, \sqrt t])=F_X(\sqrt t)-F_X(-\sqrt t)$$

Na podstawie gęstości znajdujemy wzór dystrybuanty obliczając 

$$f(t)=\int_{-2}^{x}\frac{1}{4}dt=\frac{1}{4}*(x+2)\\$$

Zatem:

$$
F_{X}(x)
 = \left\{ \begin{array}{ll}
\frac{1}{4}*(x+2) & \text{gdy } x \in [-2,2]\\
0 & \text{gdy } x \notin [-2,2]
\end{array} \right.
$$

Wyznaczamy gęstość rozkładu zmiennej losowej Y:

$\phi(x)=x^2$

$y=x^2$

$x= \sqrt y$

$h(x)= \sqrt y$

$h'(x)= \frac{1}{2}*\frac{1}{\sqrt y}$

$g(y)=f(h(y))*|h'(y)|*\chi_{\phi \in [-2,2]}{(y)}=\frac{1}{4}*\frac{1}{2}*\frac{1}{\sqrt y}, y \in [0,4]$
