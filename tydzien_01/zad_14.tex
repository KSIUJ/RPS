\subsection{Zadanie 14}

\subsubsection*{Treść}
Przy okrągłym stole przydzielono miejsca w sposób losowy 10-ciu osobom, wśród tych osób jest rodzina (rodzice (dwoje) i trójka dzieci). Ile jest sposobów przydziału miejsc przy okrągłym stole w taki sposób, aby dzieci siedziały bezpośrednio między rodzicami?


\subsubsection*{Rozwiązanie}
Rodzina przy stole musi siedzieć w okreslonej kolejności RDDDR (R-rodzic, D-dziecko).
Możemy ich ustawić na $3!$ (na tyle sposobów dzieci) $* 2!$ (na tyle sposobów rodzice)  $= 12$ sposobów. 
Następnie traktując rodzinę jako jeden element permutujemy ją wraz z pozostałymi osobami siedzącymi przy
stole. Możemy to zrobić na $6!$ sposobów. Wszystkich sposobów rozsadzenia ludzi przy stole jest:
$$
2! * 3! * 6! = 8 640
$$

