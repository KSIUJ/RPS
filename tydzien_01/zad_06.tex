\subsection{Zadanie 6}

\subsubsection*{Treść}
W puli znajdują się bile: 4 czarne, 2 niebieskie i jedna biała. Losujemy dwie bile (bez zwracania). Oblicz prawdopodobieństwo zdarzenia, polegającego na wyciągnięciu jednej bili białej i jednej czarnej. 


\subsubsection*{Rozwiązanie}
$$ |\Omega|= \binom{7}{2}=21 $$

Liczba wystąpień zdarzenia:

$$ |A| = 1* \binom{4}{1}=4 $$

Prawdopodobieństwo zdarzenia:

$$ P(A)=\frac{|A|}{|\Omega|}=\frac{4}{21} $$

